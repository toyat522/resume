\documentclass[
	%a4paper,
	10pt,
]{resume}

\usepackage{ebgaramond}
\usepackage[hidelinks]{hyperref}

\name{Toya Takahashi}

\address{(907)~$\cdot$~538~$\cdot$~1519 \\ \href{mailto:toyat@mit.edu}{\underline{toyat@mit.edu}} \\ \href{https://www.linkedin.com/in/toya-takahashi/}{\underline{linkedin/toya-takahashi}} \\ \href{https://github.com/toyat522}{\underline{github/toyat522}} \\ Cambridge, MA}

\begin{document}

\begin{rSection}{Education}
	
	\textbf{Massachusetts Institute of Technology (M.I.T.)} \hfill \textit{Expected in May 2026} \\ 
	B.S. in Electrical Engineering and Computer Science | GPA: 5.0/5.0 \\
    Relevant Coursework: Robotics, Autonomous Navigation, Algorithms, Controls, Digital Systems, Electrical Circuits
	
\end{rSection}

\begin{rSection}{Experience}

	\begin{rSubsection}{MIT Arcturus Robotics}{September 2022 - Present}{Autonomy Software Team Co-Lead}{Cambridge, MA}
    \item Leading a team of approximately 20 students to develop an Autonomous Surface Vehicle (ASV) autonomy stack in C++ and Python, utilizing Robot Operating System (ROS) 2 and MOOS-IvP middlewares.
    \item 
    \item 
	\end{rSubsection}

	\begin{rSubsection}{NVIDIA}{May 2024 - August 2024}{Systems Software Engineering Intern}{Santa Clara, CA}
    \item Lorem ipsum dolor sit amet, consectetur adipiscing elit. Donec a diam lectus.
    \item Donec ut libero sed arcu vehicula ultricies a non tortor. Lorem ipsum dolor sit amet, consectetur adipiscing elit.
    \item Aliquam at massa ipsum. Quisque bash bibendum purus convallis nulla ultrices ultricies.
	\end{rSubsection}

	\begin{rSubsection}{MIT EECS Department}{February 2024 - May 2024}{Lab Assistant, ``Computation Structures"}{Cambridge, MA}
    \item Assisted MIT undergraduate students with lab assignments for an introductory computer architecture and operating systems course.
	\end{rSubsection}

	\begin{rSubsection}{MIT Sea Grant College}{January 2023 - May 2024}{Undergraduate Researcher}{Cambridge, MA}
    \item Modeled an oyster farm simulation environment using the Gazebo and ArduPilot SITL (Software in the Loop) simulators to facilitate testing and validating an ASV autonomy stack.
    \item Wrote Unified Robot Description Format (URDF) and Simulation Description Format (SDF) files of ships, oyster baskets, and ocean waves to generate models with accurate dynamics, ensuring realistic interactions and controller feedback withing the simulation.
    \item Designed and implemented cross-hull electrical wiring for microcontrollers, stepper motors, and sensors.
    \end{rSubsection}

	\begin{rSubsection}{MIT Media Lab: Signal Kinetics}{June 2023 - December 2023}{Undergraduate Researcher}{Cambridge, MA}
    \item Operated the UR5e robot arm to collect millimeter wave radar, OptiTrack motion capture, and camera data, contributing to the development of a robot capable of searching for and retrieving hidden items.
    \item Wrote C++ and Python scripts using data analysis packages such as NumPy and Matplotlib to construct a machine learning dataset of simulated and robot-collected radar images.
	\end{rSubsection}

	\begin{rSubsection}{FIRST Robotics Competition}{September 2018 - May 2022}{Team President}{Anchorage, AK}
    \item Winner of the 2021 FIRST Tech Challenge Dean's List Award, awarded to 20 out of over 121,000 FIRST participants internationally for demonstrating leadership and achieving technical expertise in robotics.
    \item Developed path-planning algorithms for autonomous robots using dead reckoning, visual odometry, wheel encoders, PID controllers, pure pursuit controllers, and color segmentation using OpenCV.
	\end{rSubsection}

\end{rSection}

\begin{rSection}{Technical Skills}

	\begin{tabular}{@{} >{\bfseries}l @{\hspace{6ex}} l @{}}
		Computer Languages & Python, C/C++, CUDA, Java, MATLAB, SystemVerilog, RISC-V Assembly \\
		Tools & Git, Docker, Linux, Robot Operating System (ROS), Computer-Aided Design (CAD), Simulink
	\end{tabular}

\end{rSection}

\end{document}
