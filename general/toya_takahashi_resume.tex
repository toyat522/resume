\documentclass[
	%a4paper,
	10pt,
]{resume}

\usepackage{ebgaramond}
\usepackage[hidelinks]{hyperref}

\name{Toya Takahashi}

\address{(907)~$\cdot$~538~$\cdot$~1519 \\ \href{mailto:toyat@mit.edu}{\underline{toyat@mit.edu}} \\ \href{https://www.linkedin.com/in/toya-takahashi/}{\underline{linkedin/toya-takahashi}} \\ \href{https://github.com/toyat522}{\underline{github/toyat522}} \\ Cambridge, MA}

\begin{document}

\begin{rSection}{Education}

	\textbf{Massachusetts Institute of Technology (M.I.T.)} \hfill \textit{Expected in May 2026} \\ 
	B.S. in Electrical Engineering and Computer Science | GPA: 5.0/5.0 \\
    %Relevant Coursework: Robotics, Autonomous Navigation, Algorithms, Controls, Digital Systems, Electrical Circuits
    Relevant Coursework: Robotics, Algorithms, Controls, Computation Structures, Electrical Circuits
\end{rSection}

\begin{rSection}{Experience}

	\begin{rSubsection}{MIT Arcturus Robotics}{September 2022 - Present}{Autonomy Software Team Co-Lead}{Cambridge, MA}
    \item Leading a software team of approximately 20 students in developing an Autonomous Surface Vehicle (ASV) autonomy stack using C++ and Python with Robot Operating System (ROS 2) middleware.
    \item Developed an algorithm to overlay clustered LiDAR point cloud on the camera frame for matching obstacles with detected objects.
    \item Implemented an Extended Kalman Filter to fuse GPS and IMU data for global robot localization with centimeter-level accuracy.
    \item Created a visual navigation algorithm for buoy traversal, integrating the YOLOv5 object detection model with a PID controller.
    \end{rSubsection}

	\begin{rSubsection}{NVIDIA}{May 2024 - August 2024}{Systems Software Engineering Intern}{Santa Clara, CA}
    \item Enhanced the performance of an end-to-end robot manipulator object-following workflow by tripling throughput and improving the stability of object pose estimations detected by a deep neural network.
    \item Implemented and wrote unit tests for a suite of ROS nodes for post-processing a stream of poses through averaging, stability analysis, outlier detection, and Kalman filtering.
    \item Developed and optimized a CUDA-accelerated alpha compositing ROS node, enabling efficient image blending directly on the GPU without redundant CPU-GPU memory transfers.
    \item Calibrated camera intrinsics using ArUco and ChArUco boards to minimize reprojection error for improved 3D scene mapping accuracy.
    \end{rSubsection}

	\begin{rSubsection}{MIT EECS Department}{February 2024 - May 2024}{Lab Assistant, ``Computation Structures"}{Cambridge, MA}
    \item Assisted undergraduate students with lab assignments for an introductory computer architecture and operating systems course.
	\end{rSubsection}

	\begin{rSubsection}{MIT Sea Grant College}{January 2023 - May 2024}{Undergraduate Researcher}{Cambridge, MA}
    \item Modeled an oyster farm simulation environment in the Gazebo robotics simulator to test and validate an ASV autonomy stack.
    \item Created Unified Robot Description Format (URDF) and Simulation Description Format (SDF) files for ships, oyster baskets, and ocean waves to generate realistic simulation models.
    \item Designed and built a cross-hull electrical wiring system for integrating microcontrollers, stepper motors, and sensors.
    \end{rSubsection}

	\begin{rSubsection}{MIT Media Lab: Signal Kinetics}{June 2023 - December 2023}{Undergraduate Researcher}{Cambridge, MA}
    \item Operated the UR5e robot arm to collect millimeter-wave radar, OptiTrack motion capture, and camera data, contributing to the development of a robot designed to search for and retrieve hidden items.
    \item Wrote C++ and Python scripts using data analysis packages such as NumPy and Matplotlib to construct a machine learning dataset of simulated and robot-collected radar images.
	\end{rSubsection}

	\begin{rSubsection}{MIT Code for Good}{October 2023 - February 2023}{Team Leader}{Cambridge, MA}
	\item Led a team of 6 to develop a secure web application to collect and visualize client data on behalf of Thrive and Support Advocacy, a nonprofit organization supporting youth and adults with developmental disabilities.
	\item Engineered user-friendly web interfaces with ReactJS for uploading survey results to the server and visualizing collected data.
    \item Integrated front-end UI with back-end authentication and data handling systems using MongoDB, ExpressJS, and NodeJS.
	\end{rSubsection}

	%\begin{rSubsection}{FIRST Robotics Competition}{September 2018 - May 2022}{Team President}{Anchorage, AK}
    %\item Awarded the 2021 FIRST Tech Challenge Dean's List for demonstrating leadership and technical expertise. Recognized among a global pool of over 121,000 participants.
    %\end{rSubsection}

\end{rSection}

\begin{rSection}{Technical Skills}

	\begin{tabular}{@{} >{\bfseries}l @{\hspace{6ex}} l @{}}
		%Computer Languages & Python, C/C++, NumPy, CUDA, JavaScript, MATLAB, SystemVerilog, RISC-V Assembly \\
		Computer Languages & Python, C/C++, NumPy, CUDA, JavaScript, MATLAB, RISC-V Assembly \\
		Tools & Git, Docker, Linux, Robot Operating System (ROS), Computer-Aided Design (CAD), Simulink
	\end{tabular}

\end{rSection}

\end{document}
